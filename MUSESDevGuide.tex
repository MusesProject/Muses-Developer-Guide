\documentclass[a4paper,11pt]{book}
\usepackage[T1]{fontenc}
\usepackage[utf8]{inputenc}
\usepackage{lmodern}
\usepackage{url}
\usepackage[UKenglish]{babel}
\usepackage[UKenglish]{isodate}

\addto{\captionsspanish}{\renewcommand{\chaptername}{Chapter}}
\addto{\captionsspanish}{\renewcommand{\appendixname}{Appendix}}
\addto{\captionsspanish}{\renewcommand{\indexname}{Appendix}}
\addto{\captionsspanish}{\renewcommand{\contentsname}{Appendix}}
\addto{\captionsspanish}{\renewcommand{\bibname}{Bibliography}}

\title{MUSES Developer Guide}
\author{Yasir Ali, Sergio Zamarripa, Paloma de las Cuevas Delgado, ...}
\cleanlookdateon

\begin{document}

\maketitle
\tableofcontents

\chapter{Introduction}
\label{ch:intro}

This guide describes everything a developer needs to know to start developing for the MUSES system.

The MUSES System \cite{deliverable21} has been developed following a client-server architecture.
 
The client or device side is related to the end user, usually an employee who uses a mobile or portable, and possibly
personal, device to access enterprise resources. From the enterprise security point of view, the system should prevent the user from using the device incorrectly. Therefore, MUSES monitors the user's context and behaviour, and controls their actions, allowing or forbidding them depending on those policies.

The MUSES server side is controlled by an enterprise security manager, the Chief Security Officer (CSO), who defines the security policies to be considered in the system. The security policies are used by the MUSES server to identify which behaviour is allowed and which one is not. The server then receives, stores, and processes all the gathered information from users' devices. After that, it analyses the data, performing a real-time risk and trust analysis. In addition, the system detects policy violations through event correlation techniques, adapting to changes in the environment, as well as non-secure user behaviours.

\section{MUSES on Github}
\label{sec:musesgit}

All the MUSES system code is available at \url{https://github.com/MusesProject}. That is the Github organisation page for MUSES, and it contains the following repositories:

\begin{description}
  \item[Muses-Security-Quiz]
  \item[MusesCommon] 
  \item[MusesServer] This repository contains all the files in the Java-Maven MUSES server project.
  \item[MusesClient] Repository containing the first prototype of the MUSES client, developed for Android.
  \item[MusesSituationPredictionStudy]
  \item[MusesClientIOS] Repository containing the first prototype of the MUSES client, developed for iOS.
  \item[MusesAwareLibIOS] A library that can be used by applications to retrieve information relevant for MUSES.
  \item[MusesAwareApp] It contains the files of an Android project which consists of a MUSES-aware application. A MUSES-Aware application is an application that uses the MUSES API in order to notify MUSES of user interactions and adapt its behaviour based on MUSES provided commands/security decisions \cite{deliverable24}.
  \item[MusesAwareAppTest] Its purpose is testing the MUSES-Aware application.
\end{description}

This guide is structured as follows. First, Chapters \ref{ch:client} and \ref{ch:server} detail how to build MUSES, in order to be able to star developing and testing the system, either you want to develop for the client (Chapter \ref{ch:client}), the server (Chapter \ref{ch:server}), or both. Then, Chapter \ref{ch:sensors} specifies how to integrate new sensors to the MUSES client, for better monitoring user's actions and making the \textit{context\footnote{Any information that can be used to characterize the situation of the user.\cite{deliverable61}} observation} more complete. Finally, Chapter \ref{ch:installmuses} enumerates the steps to install MUSES in a real company environment.

\chapter{Building MUSES Client}
\label{ch:client}

The first prototype of MUSES has been developed for Android. This chapter describes the needed tools, as well as building instructions, for developing inside the MUSES client. For instructions about how to install the whole system, please refer to Chapter \ref{ch:installmuses}.

\section{Installing Android SDK}
\label{sec:ADT}

As we will use Eclipse for MUSES development, the first step is to install the Android Development Tools (ADT) plugin. The requiremenst are \cite{adt:site}:

\begin{itemize}
  \item Eclipse 3.7.2 (Indigo) or greater.
  \item Java Platform (JDK) 6.
  \item Eclipse Java Development Tools (JDT) plugin. To install this pluging, in Eclipse go to \textit{Help > Install New Software...}, and make sure the \textit{Eclipse <youreclipsename> Update Site} is available by clicking in \textit{Find more software by working with the Available Software Sites preferences.}. Then select it and on the package list, look for \textit{Programming Languages > Eclipse Java Development Tools}. Finally, select Eclipse Java Development Tools and click \textit{Next} and \textit{Finish}. Eclipse will need to be restarted.
\end{itemize}

Now we can install the ADT plugin. This can be done by selecting again \textit{Help > Install New Software...} and then looking at this reposiroty \url{https://dl-ssl.google.com/android/eclipse/}. After that, selecting \textit{Developer Tools} from the package list for installing it (by clicking \textit{Next} and \textit{Finish}), and then restart Eclipse one more time.

\chapter{Building MUSES Server}
\label{ch:server}


\chapter{Creating and integrating new sensors}
\label{ch:sensors}


\chapter{Installing MUSES}
\label{ch:installmuses}

\bibliographystyle{abbrv}
\bibliography{MUSESDevGuide}

\end{document}
